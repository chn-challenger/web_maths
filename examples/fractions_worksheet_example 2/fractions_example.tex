%\documentclass{article}
%\usepackage[math]{iwona}
%\usepackage[fleqn]{amsmath}
%\usepackage{scrextend}
%\changefontsizes[20pt]{17pt}
%\usepackage[margin=0.5in]{geometry}
%\pagenumbering{gobble}
%\usepackage{setspace}
%\renewcommand{\baselinestretch}{2.5}

%\usepackage{amsmath,amssymb}

%\allowdisplaybreaks[1]
%\newcommand{\centerdia}[1]{\makebox[2in]{\includegraphics{#1}}}
%
%\begin{document}
%\section*{\centerline{Fraction Worksheet 2}}
%\vspace{10 mm}
%\begin{align*}
%1.\hspace{10pt}&2\frac{3}{4} + 4 \frac{2}{7} &12.\hspace{10pt} &4\frac{2}{5} + 3\frac{5}{9}\\[2em]
%3.\hspace{10pt}&2\frac{3}{4}  + 2314 \frac{12}{17} &11.\hspace{10pt}&4\frac{2}{5} + 3\frac{5}{9}\\[2em]
%1.\hspace{10pt}&2\frac{3}{4} + 4 \frac{2}{7} \hspace{90pt} &12.\hspace{10pt} &4\frac{2}{5} + 3\frac{5}{9}\\[2em]
%3.\hspace{10pt}&2\frac{3}{4}  + 2314 \frac{12}{17} &11.\hspace{10pt}&4\frac{2}{5} + 3\frac{5}{9}\\[2em]
%1.\hspace{10pt}&2\frac{3}{4} + 4 \frac{2}{7} \hspace{90pt} &12.\hspace{10pt} &4\frac{2}{5} + 3\frac{5}{9}\\[2em]
%3.\hspace{10pt}&2\frac{3}{4}  + 2314 \frac{12}{17} &11.\hspace{10pt}&4\frac{2}{5} + 3\frac{5}{9}\\[2em]
%3.\hspace{10pt}&2\frac{3}{4}  + 4 \frac{2}{7}  &11.&4\frac{2}{5} + 3\frac{5}{9}
%1.\hspace{10pt}&2\frac{3}{4} + 4 \frac{2}{7} &12.\hspace{10pt} &4\frac{2}{5} + 3\frac{5}{9} & 12.\hspace{10pt} &4\frac{2}{5} + 3\frac{5}{9}\\[1em]
%1.\hspace{10pt}&2\frac{3}{4} + 4 \frac{2}{7} &12.\hspace{10pt} &4\frac{2}{5} + 3\frac{5}{9} & 12.\hspace{10pt} &4\frac{2}{5} + 3\frac{5}{9}
%1.\hspace{15pt}&1\frac{1}{2}\times4\frac{1}{4}&2.\hspace{15pt}&1\frac{1}{2}\div1\frac{2}{3}\\[2em]
%3.\hspace{15pt}&2\frac{1}{2}\times1\frac{4}{5}&4.\hspace{15pt}&1\frac{2}{3}-1\frac{1}{10}\\[2em]
%5.\hspace{15pt}&10\frac{7}{8}+3\frac{5}{6}&6.\hspace{15pt}&6\frac{1}{4}\times3\frac{1}{6}\\[2em]
%7.\hspace{15pt}&4\frac{3}{4}-1\frac{2}{3}&8.\hspace{15pt}&\frac{3}{7}\times6\frac{4}{7}\\[2em]
%9.\hspace{15pt}&4\frac{3}{5}\times6\frac{2}{3}&10.\hspace{15pt}&1\frac{2}{7}\div\frac{1}{5}\\[2em]
%11.\hspace{15pt}&\frac{2}{3}\div5\frac{7}{8}&12.\hspace{15pt}&7\frac{2}{3}+4\frac{1}{3}
%\end{align*}
%\end{document}

%
%\begin{equation*}
%\begin{aligned}
% 1.\hspace{10pt}&2\frac{3}{4} + 4 \frac{2}{7}\\[2em]
%  1.\hspace{10pt}&2\frac{3}{4} + 4 \frac{2}{7}\\[2em]
%   1.\hspace{10pt}&2\frac{3}{4} + 4 \frac{2}{7}\\
%\end{aligned}
%\hspace{40pt}
%\begin{aligned}
% 1.\hspace{10pt}&2\frac{3}{4} + 4 \frac{2}{7}\\[2em]
% 1.\hspace{10pt}&2\frac{3}{4} + 4 \frac{2}{7}\\[2em]
% 1.\hspace{10pt}&2\frac{3}{4} + 4 \frac{2}{7}\\
%\end{aligned}
%\hspace{40pt}
%\begin{aligned}
%1.\hspace{10pt}&2\frac{3}{4} + 4 \frac{2}{7}\\[2em]
%1.\hspace{10pt}&2\frac{3}{4} + 4 \frac{2}{7}\\[2em]
%1.\hspace{10pt}&2222\frac{3}{4} + 4 \frac{2}{7}\\
%\end{aligned}
%\hspace{40pt}
%\begin{aligned}
%1.\hspace{10pt}&2\frac{3}{4} + 4 \frac{2}{7}\\[2em]
%1.\hspace{10pt}&2\frac{3}{4} + 4 \frac{2}{7}\\[2em]
%1.\hspace{10pt}&2333\frac{3}{4} + 4 \frac{2}{7}\\
%\end{aligned}
%\begin{aligned}
%P_{S}   &= V_{S} \times I_{1} \times cos(\phi) \\
%&=  \times  \times cos(\phi) \\
%\therefore \
%cos(\phi)_{1A}  &= \frac{}{ \times } \\
%&=  \\
%I_{S}           &= I_{1} \sqrt{1+THD^{2}} \\
%\therefore \ 
%THD     &= \frac{I_{S}}{I_{1}} - 1\\    
%&= \frac{}{} - 1 \\
%&=  \\
%df      &= cos(\phi) \times \frac{1}{\sqrt{1+THD^{2}}} \\               
%&=  \times \frac{1}{\sqrt{1+^{2}}} \\
%&= \\
%\end{aligned}
%\end{equation*}

%\documentclass{article}
%\usepackage[math]{iwona}
%\usepackage[fleqn]{amsmath}
%\usepackage{scrextend}
%\changefontsizes[20pt]{17pt}
%\usepackage[a4paper, left=0.7in,right=0.7in,top=1in,bottom=1in]{geometry}
%\pagenumbering{gobble}
%\usepackage{fancyhdr}
%\renewcommand{\headrulewidth}{0pt}
%\pagestyle{fancy}
%\lfoot{FRC-AB12341Q\quad \textcopyright\, Joe Zhou, 2015}
%\rfoot{Jack Lu}
%
%
%\begin{document}
%	\section*{\centerline{Fraction Worksheet Test Solutions}}
%	\vspace{10 mm}
%	\begin{align*}
%&1.\hspace{15pt}1\frac{1}{2}\times4\frac{1}{4}&&2.\hspace{15pt}1\frac{1}{2}\div1\frac{2}{3}\\[2em]
%&3.\hspace{15pt}2\frac{1}{2}\times1\frac{4}{5}&&4.\hspace{15pt}1\frac{2}{3}-1\frac{1}{10}\\[2em]
%&5.\hspace{15pt}10\frac{7}{8}+3\frac{5}{6}&&6.\hspace{15pt}6\frac{1}{4}\times3\frac{1}{6}\\[2em]
%&7.\hspace{15pt}4\frac{3}{4}-1\frac{2}{3}&&8.\hspace{15pt}\frac{3}{7}\times6\frac{4}{7}\\[2em]
%&9.\hspace{15pt}4\frac{3}{5}\times6\frac{2}{3}&&10.\hspace{15pt}1\frac{2}{7}\div\frac{1}{5}
%	\end{align*}
%\end{document}

%\documentclass{article}
%\usepackage[math]{iwona}
%\usepackage[fleqn]{amsmath}
%\usepackage{scrextend}
%\changefontsizes[20pt]{17pt}
%\usepackage[a4paper, left=0.7in,right=0.7in,top=1in,bottom=1in]{geometry}
%\pagenumbering{gobble}
%\usepackage{fancyhdr}
%\renewcommand{\headrulewidth}{0pt}
%\pagestyle{fancy}
%\lfoot{FRA-UW156659Q\quad \textcopyright\, Joe Zhou, 2015}
%\rfoot{\textit{student:}\quad Jack Lu}
%\begin{document}
%	\section*{\centerline{Fraction Worksheet 1}}
%	\vspace{10 mm}
%	\begin{align*}
%&1.\hspace{15pt}3+\frac{8}{6x-11}=11&&2.\hspace{15pt}2\left(7\left(x-3\right)+24\right)=132\\[2em]
%&3.\hspace{15pt}34-\frac{2x+34}{4}=23&&4.\hspace{15pt}8+\frac{10}{41-9x}=10\\[2em]
%&5.\hspace{15pt}9\left(88-8x\right)+6=438&&6.\hspace{15pt}3\left(41-\frac{32+x}{2}\right)=72\\[2em]
%&7.\hspace{15pt}4\left(9-\frac{31-x}{4}\right)=8&&8.\hspace{15pt}3\left(\frac{240}{38-x}-3\right)=15\\[2em]
%&9.\hspace{15pt}5\left(143-5\left(x+9\right)\right)=240&&10.\hspace{15pt}6\left(36-\frac{x}{9}\right)-14=196\\[2em]
%&11.\hspace{15pt}9\left(20-\frac{81}{x}\right)+28=127&&12.\hspace{15pt}4\left(\frac{98}{8+x}+9\right)=64
%	\end{align*}
%\end{document}


%\documentclass{article}
%\usepackage[math]{iwona}
%\usepackage[fleqn]{amsmath}
%\usepackage{scrextend}
%\changefontsizes[20pt]{17pt}
%\usepackage[a4paper, left=0.7in,right=0.7in,top=1in,bottom=1in]{geometry}
%\pagenumbering{gobble}
%\usepackage{fancyhdr}
%\renewcommand{\headrulewidth}{0pt}
%\pagestyle{fancy}
%\lfoot{LEN-UM901121Q\quad \textcopyright\, Joe Zhou, 2015}
%\rfoot{\textit{student:}\quad Billy Thomas}
%\begin{document}
%	\section*{\centerline{Linear Equations Worksheet 1}}
%	\vspace{10 mm}
%	\begin{align*}
%		&1.\hspace{15pt}48+6\left(44+\frac{x}{3}\right)=318&&2.\hspace{15pt}23+\frac{85}{2x+9}=28\\[2em]
%		&3.\hspace{15pt}2\left(\frac{x}{9}+14\right)-26=4&&4.\hspace{15pt}5\left(36+\frac{x}{7}\right)-9=176\\[2em]
%		&5.\hspace{15pt}7\left(5\left(x+32\right)+38\right)=1631&&6.\hspace{15pt}5+3\left(50-\frac{x}{5}\right)=152\\[2em]
%		&7.\hspace{15pt}\frac{1185}{7\left(42-x\right)-29}=5&&8.\hspace{15pt}4\left(\frac{24+x}{2}+23\right)=144\\[2em]
%		&9.\hspace{15pt}257-4\left(7+9x\right)=13&&10.\hspace{15pt}\frac{3\left(18-x\right)+3}{6}=6\\[2em]
%		&11.\hspace{15pt}2\left(\frac{148}{x+27}-3\right)=2&&12.\hspace{15pt}9\left(28-6\left(x-7\right)\right)=198\\[2em]
%		&13.\hspace{15pt}3\left(\frac{x+23}{5}+50\right)=165&&14.\hspace{15pt}\frac{5\left(28-x\right)+15}{3}=45\\[2em]
%		&15.\hspace{15pt}3\left(\frac{72}{x-2}+21\right)=90&&16.\hspace{15pt}2\left(4+\frac{27}{x}\right)-4=10
%	\end{align*}
%\end{document}



\documentclass{article}

\usepackage[landscape,left=0.7in,right=0.7in,top=1in,bottom=1in]{geometry}
\usepackage{amsmath,amssymb}

\begin{document}
	\begin{equation*}
\begin{aligned}
P_{S}   &= V_{S} \times I_{1} \times cos(\phi) \\
&=  \times  \times cos(\phi) \\
\therefore \
cos(\phi)_{1A}  &= \frac{}{ \times } \\
&=  \\
I_{S}           &= I_{1} \sqrt{1+THD^{2}} \\
\therefore \ 
THD     &= \frac{I_{S}}{I_{1}} - 1\\    
&= \frac{}{} - 1 \\
&=  \\
df      &= cos(\phi) \times \frac{1}{\sqrt{1+THD^{2}}} \\               
&=  \times \frac{1}{\sqrt{1+^{2}}} \\
&=      
\end{aligned}
	\begin{aligned}
	P_{S}   &= V_{S} \times I_{1} \times cos(\phi) \\
	&=  \times  \times cos(\phi) \\
	\therefore \
	cos(\phi)_{1A}  &= \frac{}{ \times } \\
	&=  \\
	I_{S}           &= I_{1} \sqrt{1+THD^{2}} \\
	\therefore \ 
	THD     &= \frac{I_{S}}{I_{1}} - 1\\    
	&= \frac{}{} - 1 \\
	&=  \\
	df      &= cos(\phi) \times \frac{1}{\sqrt{1+THD^{2}}} \\               
	&=  \times \frac{1}{\sqrt{1+^{2}}} \\
	&=      
	\end{aligned}
	\begin{aligned}
	P_{S}   &= V_{S} \times I_{1} \times cos(\phi) \\
	&=  \times  \times cos(\phi) \\
	\therefore \
	cos(\phi)_{1A}  &= \frac{}{ \times } \\
	&=  \\
	I_{S}           &= I_{1} \sqrt{1+THD^{2}} \\
	\therefore \ 
	THD     &= \frac{I_{S}}{I_{1}} - 1\\    
	&= \frac{}{} - 1 \\
	&=  \\
	df      &= cos(\phi) \times \frac{1}{\sqrt{1+THD^{2}}} \\               
	&=  \times \frac{1}{\sqrt{1+^{2}}} \\
	&= \\
	\end{aligned}
	\begin{aligned}
	P_{S}   &= V_{S} \times I_{1} \times cos(\phi) \\
	&=  \times  \times cos(\phi) \\
	\therefore \
	cos(\phi)_{1A}  &= \frac{}{ \times } \\
	&=  \\
	I_{S}           &= I_{1} \sqrt{1+THD^{2}} \\
	\therefore \ 
	THD     &= \frac{I_{S}}{I_{1}} - 1\\    
	&= \frac{}{} - 1 \\
	&=  \\
	df      &= cos(\phi) \times \frac{1}{\sqrt{1+THD^{2}}} \\               
	&=  \times \frac{1}{\sqrt{1+^{2}}} \\
	&= \\
	\end{aligned}
	\end{equation*}
	\begin{equation*}
	\begin{aligned}
	P_{S}   &= V_{S} \times I_{1} \times cos(\phi) \\
	&=  \times  \times cos(\phi) \\
	\therefore \
	cos(\phi)_{1A}  &= \frac{}{ \times } \\
	&=  \\
	I_{S}           &= I_{1} \sqrt{1+THD^{2}} \\
	\therefore \ 
	THD     &= \frac{I_{S}}{I_{1}} - 1\\    
	&= \frac{}{} - 1 \\
	&=  \\
	df      &= cos(\phi) \times \frac{1}{\sqrt{1+THD^{2}}} \\               
	&=  \times \frac{1}{\sqrt{1+^{2}}} \\
	&=      
	\end{aligned}
	\begin{aligned}
	P_{S}   &= V_{S} \times I_{1} \times cos(\phi) \\
	&=  \times  \times cos(\phi) \\
	\therefore \
	cos(\phi)_{1A}  &= \frac{}{ \times } \\
	&=  \\
	I_{S}           &= I_{1} \sqrt{1+THD^{2}} \\
	\therefore \ 
	THD     &= \frac{I_{S}}{I_{1}} - 1\\    
	&= \frac{}{} - 1 \\
	&=  \\
	df      &= cos(\phi) \times \frac{1}{\sqrt{1+THD^{2}}} \\               
	&=  \times \frac{1}{\sqrt{1+^{2}}} \\
	&=      
	\end{aligned}
	\begin{aligned}
	P_{S}   &= V_{S} \times I_{1} \times cos(\phi) \\
	&=  \times  \times cos(\phi) \\
	\therefore \
	cos(\phi)_{1A}  &= \frac{}{ \times } \\
	&=  \\
	I_{S}           &= I_{1} \sqrt{1+THD^{2}} \\
	\therefore \ 
	THD     &= \frac{I_{S}}{I_{1}} - 1\\    
	&= \frac{}{} - 1 \\
	&=  \\
	df      &= cos(\phi) \times \frac{1}{\sqrt{1+THD^{2}}} \\               
	&=  \times \frac{1}{\sqrt{1+^{2}}} \\
	&= \\
	\end{aligned}
	\end{equation*}
\end{document}
